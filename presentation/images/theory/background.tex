\documentclass[tikz, border=0pt]{standalone}
\usepackage{siunitx}
\usepackage{physics}
\usepackage[compat=1.1.8]{tikz-feynman}
\usetikzlibrary{decorations,decorations.markings,decorations.text}

\begin{document}

\def\picscale{1.5}
\def\nodescale{0.5}
\def\labscale{0.5}

\begin{tikzpicture}
    \begin{feynman}[every blob={/tikz/inner sep=0pt, /tikz/minimum size=3mm}, scale=\picscale, transform shape]
        \vertex [anchor=center, scale=\nodescale]                         (g1) {\color{red}\( g \)};
        \vertex [below right=of g1, anchor=center, dot, scale=\nodescale] (v1) {};
        \vertex [below left =of v1, anchor=center, scale=\nodescale]      (g2) {\color{red}\( g \)};
        \vertex [right      =of v1, anchor=center, dot, scale=\nodescale] (v2) {};
        \vertex [above right=of v2, anchor=center, dot, scale=\nodescale] (v3) {};
        \vertex [below right=of v2, anchor=center, dot, scale=\nodescale] (v4) {};
        \vertex [above right=of v3, anchor=center, scale=\nodescale]      (b1) {\color{red}\( b \)};
        \vertex [below right=of v3, anchor=south , scale=\nodescale]      (w1) {\color{red}\( W^{+} \)};
        \vertex [above right=of v4, anchor=north , scale=\nodescale]      (w2) {\color{red}\( W^{-} \)};
        \vertex [below right=of v4, anchor=center, scale=\nodescale]      (b2) {\color{red}\( \bar{b} \)};

        \diagram*{
            (g1) -- [gluon] (v1),
            (g2) -- [gluon] (v1),
            (v1) -- [gluon] (v2),
            (v4) -- [fermion, edge label=\( \bar{t} \), scale=\labscale] (v2) -- [fermion, edge label=\( t \), scale=\labscale] (v3),
            (w1) -- [boson] (v3) -- [fermion] (b1),
            (b2) -- [fermion] (v4) -- [boson] (w2),
        };
    \end{feynman}
\end{tikzpicture}

\end{document}