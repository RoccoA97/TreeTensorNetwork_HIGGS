% Paper template for Advanced Physics Laboratory Course
% Suggestion: it is better to compile it directly on overleaf, because the subfigure settings can arise problems
% Made by Alice Pagano and Francesca Dodici
% Modified by Rocco Ardino





%%%%%%%%%%%%%%%%%%%%%%%%%%%%%%%%%%%%%%%%%%%%%%%%%%%%%%%%%%%%%%%%%%%%%%%%%%%%%%%%
% LOAD PACKAGES
%%%%%%%%%%%%%%%%%%%%%%%%%%%%%%%%%%%%%%%%%%%%%%%%%%%%%%%%%%%%%%%%%%%%%%%%%%%%%%%%
%% general packages
\usepackage{graphicx}
\usepackage{wrapfig}
\usepackage{color}
\usepackage{latexsym,amsmath}
\usepackage{physics}
\usepackage{chemformula}
\usepackage{tabularx}
\usepackage{float}
\usepackage{siunitx}
\usepackage{amssymb}
\usepackage{slashed}
%\usepackage[caption=false]{subfig}
\usepackage[caption=false]{subfig}
% \usepackage{subcaption}
\usepackage[export]{adjustbox}
\usepackage{multirow}
\usepackage[margin=2cm]{geometry}
\usepackage{enumitem}

%% listing packages (for codes)
\usepackage{xcolor}
\usepackage{listings}
\usepackage{framed}
\usepackage{inconsolata} % To change the listing font
\usepackage{algpseudocode}
\usepackage{algorithm}

%% standalone package
\usepackage[mode=buildnew]{standalone}

%% hyperlink package
\definecolor{linkcolor}{rgb}{0,0,0.65} %hyperlink
\definecolor{linescolor}{rgb}{0.65,0.16,0.16}
\definecolor{cool}{RGB}{49,54,149}
\definecolor{hot}{RGB}{165,0,38}
\usepackage[pdftex, colorlinks=true, pdfstartview=FitV, linkcolor=linescolor, citecolor=linescolor, urlcolor=linkcolor, hyperindex=true, hyperfigures=true]{hyperref}

%% page settings and formatting
\usepackage{fancyhdr}
% \usepackage{ragged2e}

% TODOs package
\usepackage{todonotes}

%% tikz
\usepackage{tikz}

%% tikz libraries
\usetikzlibrary{decorations,decorations.markings,decorations.text}
\usetikzlibrary{patterns}
%%%%%%%%%%%%%%%%%%%%%%%%%%%%%%%%%%%%%%%%%%%%%%%%%%%%%%%%%%%%%%%%%%%%%%%%%%%%%%%%





%%%%%%%%%%%%%%%%%%%%%%%%%%%%%%%%%%%%%%%%%%%%%%%%%%%%%%%%%%%%%%%%%%%%%%%%%%%%%%%%
% SETTINGS AND NEW COMMANDS
%%%%%%%%%%%%%%%%%%%%%%%%%%%%%%%%%%%%%%%%%%%%%%%%%%%%%%%%%%%%%%%%%%%%%%%%%%%%%%%%
%% fancyhdr header and footer settings
\pagestyle{fancyplain}
\fancyhf{}
\fancyfoot[R]{TTN supervised classifier for HEP - Page \textbf{\thepage}}
\fancyfoot[L]{Quantum Information and Computing 2020/2021}
%\fancyhead[L]{\textbf{Advanced Physics Laboratory Report}}
\fancyhead[L]{\ifnum\value{section}>0\nouppercase{\textbf{\leftmark}\fi}}
\fancyhead[R]{R. Ardino - A. Valente}
\renewcommand{\headrulewidth}{0.2pt}
\renewcommand{\footrulewidth}{0.1pt}

%% indentation space
\setlength\parindent{9pt}
% \linespread{1.13} % 11pt, highly spaced
\linespread{1.05} % 11pt, lowly spaced
% \linespread{0.956} % 0.956} % if using 11pt font

%% section style
%Redefine \thesubsection as \thesection.\alph{subsection}. (\alph replaces the default \arabic; you could also choose, e.g., \Alph, \roman, and \Roman.)
\renewcommand{\thesection}{\textbf{\Roman{section}}}
\renewcommand{\thesubsection}{\textbf{\arabic{subsection}}}
\renewcommand{\thesubsubsection}{\textbf{\Alph{subsubsection}}}
\renewcommand{\theparagraph}{\bf\sffamily\alph{paragraph}}

\makeatletter
\renewcommand{\paragraph}{% standard vertical spacing: 3.25ex
    \@startsection{paragraph}{4}%
    {\z@}{2.50ex \@plus 1ex \@minus .2ex}{-1em}%
    {\bf\sffamily}%
}
\makeatother


%\renewcommand{\thesection}{\textbf{\arabic{section}}}
%\renewcommand{\thesubsection}{\thesection.\textbf{\arabic{subsection}}}
%\renewcommand{\thesubsubsection}{\textbf{\thesection.\arabic{subsection}.\arabic{subsubsection}}}

%% figure and table style settings
% creano casini
% \renewcommand{\tablename}{\textbf{TAB.}}
% \renewcommand{\thetable}{\textbf{\arabic{table}}}
% \renewcommand{\figurename}{\textbf{FIG.}}
% \renewcommand{\thefigure}{\textbf{\arabic{figure}}}

%% bibliography settings
\bibliographystyle{aipnum4-1}
\setcitestyle{numbers,square}

%% figure caption settings (in bold)
\captionsetup[figure]{
    % justification=justified,
    % labelsep=newline,
    % justification=raggedright,
    % format=hang,
    % justification=RaggedRight,
    % singlelinecheck=off,
    labelfont={bf},
    labelformat={default},
    labelsep=period,
    name={FIG.}%
}
% \captionsetup[subfigure]{labelformat=brace}

%% colors
\definecolor{amber}{rgb}{1.0, 0.75, 0.0}
\definecolor{goldmetallic}{rgb}{0.83, 0.69, 0.22}
\definecolor{airforceblue}{rgb}{0.36, 0.54, 0.66}  %#5D8AA8
\definecolor{cobalt}{rgb}{0.0, 0.28, 0.67}         %#0047AB
\definecolor{coolblack}{rgb}{0.0, 0.18, 0.39}      %#002E63
\definecolor{dartmouthgreen}{rgb}{0.05, 0.5, 0.06} %#00693E
\definecolor{mydmg}{rgb}{0.05, 0.5, 0.06}          %#00693E
\definecolor{lava}{rgb}{0.81, 0.06, 0.13}          %#CF1020
\definecolor{myred}{rgb}{0.81, 0.06, 0.13}         %#CF1020

% itemize
\setlist[itemize]{
    noitemsep,
    topsep=5pt,
    parsep=5pt,
    partopsep=5pt
}
%%%%%%%%%%%%%%%%%%%%%%%%%%%%%%%%%%%%%%%%%%%%%%%%%%%%%%%%%%%%%%%%%%%%%%%%%%%%%%%%


\newcommand{\figref}[1]{FIG. \textbf{\ref{#1}}}
\newcommand{\eqnref}[1]{EQ. \textbf{\ref{#1}}}
\newcommand{\tabref}[1]{TAB. \textbf{\ref{#1}}}
\renewcommand{\algref}[1]{ALG. \textbf{\ref{#1}}}
\newcommand{\lstref}[1]{LST. \textbf{\ref{#1}}}


% \definecolor{codegreen}{rgb}{0,0.6,0}
% \definecolor{codegray}{rgb}{0.5,0.5,0.5}
% \definecolor{codepurple}{rgb}{0.58,0,0.82}
% \definecolor{backcolour}{rgb}{0.95,0.95,0.92}
% %Code listing style named "mystyle"
% \lstdefinestyle{mystyle}{
%   backgroundcolor=\color{backcolour},   commentstyle=\color{codegreen},
%   keywordstyle=\color{red},
%   numberstyle=\tiny\color{codegray},
%   stringstyle=\color{codepurple},
%   basicstyle=\ttfamily\footnotesize,
%   breakatwhitespace=false,         
%   breaklines=true,
%   postbreak=\mbox{\textcolor{black}{$\hookrightarrow$}\space},
%   captionpos=b,                    
%   keepspaces=true,                 
%   numbers=left,                    
%   numbersep=5pt,    -+              
%   showspaces=false,                
%   showstringspaces=false,
%   showtabs=false,                  
%   tabsize=2
% }



\definecolor{linkcolor}{rgb}{0,0,0.65}
\definecolor{shadecolor}{rgb}{0.95, 0.95, 0.95}
\definecolor{mygreen}{rgb}{0,0.6,0}
\definecolor{mygray}{rgb}{0.5,0.5,0.5}
\definecolor{mymauve}{rgb}{0.58,0,0.82}

\lstdefinestyle{python}
{
    backgroundcolor=\color{shadecolor},       % background color
    basicstyle=\ttfamily\footnotesize,        % the size of the fonts that are used for the code
    breakatwhitespace=false,                  % sets if automatic breaks should only happen at whitespace
    breaklines=true,                          % sets automatic line breaking
    postbreak=\mbox{\textcolor{black}{$\hookrightarrow$}\space},
    captionpos=b,                             % sets the caption-position to bottom
    commentstyle=\color{mygreen},             % comment style
    extendedchars=true,                       % lets you use non-ASCII characters; for 8-bits encodings only, does not work with UTF-8
    keepspaces=true,                          % keeps spaces in text, useful for keeping indentation of code (possibly needs columns=flexible)
    keywordstyle=\bfseries\color{blue},       % keyword style
    language=Python,                          % the language of the code
    numbers=left,                             % where to put the line-numbers; possible values are (none, left, right)
    numbersep=5pt,                            % how far the line-numbers are from the code
    numberstyle=\tiny\color{mygray},          % the style that is used for the line-numbers
    rulecolor=\color{black},                  % if not set, the frame-color may be changed on line-breaks within not-black text (e.g. comments (green here))
    showspaces=false,                         % show spaces everywhere adding particular underscores; it overrides 'showstringspaces'
    showstringspaces=false,                   % underline spaces within strings only
    showtabs=false,                           % show tabs within strings adding particular underscores
    stepnumber=1,                             % the step between two line-numbers. If it's 1, each line will be numbered
    stringstyle=\color{mymauve},              % string literal style
    tabsize=4,                                % sets default tabsize to 4 spaces
    title=\lstname                            % show the filename of files
}

\usepackage{pythonhighlight}

% "mystyle" code listing set
% \lstset{style=mystyle}